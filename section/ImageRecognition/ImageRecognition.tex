\section{画像認識処理}

本章では、アバポンヌ船長に特化した画像認識技術を確立するための研究について報告する。
画像認識では、物体の特徴量とよばれる変数を抜き出し、それらを元に処理を行っていく。

\subsection{Haar-Like特徴量}
Haar特徴ベースのカスケード型分類器を使った物体検出\cite{990517}は、効率的な物体検出手法である。
機械学習雨を元にした手法で、大量の正例、負例の画像からカスケード関数を最適化してく。
機械学習を終えた後、入力画像に対して手法が適用される。
矩形領域を設定し、白い領域の画素値と黒い領域の画素値を比較することで、それらの差を特徴量とする。

\subsection{アバポンヌアルゴリズム}
\subsubsection{OpenCV}
独自のアルゴリズムを開発するにあたり、openCVを使用させて貰っている。
