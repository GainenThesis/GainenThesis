%======================
\chapter{オガワマン登校分布}
%======================

重症患者には様々な爆誕方法が存在する。特に、研究室に登校する際の爆誕方法を示す表現方法は多種多様であり、目を見張るものがある。本章では、世界各地で用いられているその表現方法についてまとめる。
%==================
\section{そもそも「登校する?」という概念}
基本的に、各々の重症患者がサークルや部活、バイトや授業、竹田徹夜などで普段から生活リズムがバラバラバラバラ\footnotetext{$\sf{ (´\_ゝ`)}$バラッッッッッッッッ}である。そこで、その日にお互いが登校するかどうかを把握する上で非言語予定確認という観点から「登校する?」という概念が生まれた。たぶん。\\
 特に、オガワマンは不定期に研究室に来なくなったりするため、日々のオガワマン登校の確認という概念の分布、いわば概念分布の研究が進められた。今回は、凄腕の研究期間であるJAXA (Japan Abaponnu X-ray Abaponnu)の研究結果を示す。
\begin{figure}[H]
\centering
\includegraphics[clip,scale=0.25]{bara.jpg}
\caption{まあLINEでどんな非言語パスが来てもこれよりはマシやから何も思わない$\sf{ (´\_ゝ`)}$}
\label{bara}
\end{figure}

\section{オガワマン登校分布}
まあ一覧にするだけやけど$\sf{ (´\_ゝ`)}$笑\\
 以下は全て「登校する?」という意味である。おそらく。

\begin{itemize}
\item 生える?
\item 存在する?
\item 来る?
\item 死亡?
\item 生えてる?
\item いつ生えんの?
\item 行くべき?
\item マンゴスチン栽培は今日は無し?
\item 寿命は?
\item いま研究室?
\item 登校ポコ
\item 胃袋は?
\item きょうは農地耕作なし?
\item 剃られた?
\item 生えわたる?
\item 明日生える?→剛毛育毛剤→バンドロポン
\item ホッタイモ今日生えわたる梅酒ソーダ割り?
\item 芋掘ってる?
\item Mn登校は?(→まさか金八?)
\item ヒゲバーストに便乗して、オマンゲン寺から一時出家(→まだ入ってる?)
\item 存在?
\item 生きてる?
\item 生えまくり?
\item 梅酒?
\end{itemize}
