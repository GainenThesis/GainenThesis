
\section{周りからの総評}
オガワかをり(通称、かをり)爆誕という概念は、時をかける少女の様に2015年以来、
常に我々の頭の中を支配してきた概念である。
急激に爆誕した、そのかをりという概念について多角的に考察しようと思う。
前節で議論したように、カヲリという概念は天から一瞬にして降り注いだ概念であり、爆発的なその急成長のスペードから、どの時点で発生したかという定義が非常に困難である。
明らかにAmazonでの買い物の段階で発生したのであるが、ここで今一度入学式からのカヲリ概念時間発展を追ってみたいと思う。
\par
この章ではまず我々の入学式の時点から卒業式にかけてどのようにカヲリという概念が時間発展したかを、時系列に沿って議論する。
また、卒業式では実際に我々が追い求めていたオガワカヲリと対面することになるのであるが、それらの決定的瞬間についても報告する。
そしてそれらの結論として、カヲリがいかに皆から愛されている女性であるかを示そう。

\subsection{入学式}

まず、入学式時点のおける集合写真を図\ref{fig:H24Nyugakushiki}示そう
\footnote{併せて抑えて起きたいこの時点で形成された概念として、有名なファンヨンテと呼ばれる概念が存在し、そして少し時代が下ればマタチキスと呼ばれるホッチキス芸人が誕生する。}。
オガワカヲリという概念の生みの親である、オガワマンは最前列最左端に君臨している。
最新の解析結果によると、この時点ではオガワカヲリは、オガワマン内部にのみ存在する母親であり、竹田・又吉両名が認知するには至っていなかった様である。
そのため入学式の時点を「カヲリ内部隠遁時代」とも呼び、オガワマンによってオガワカヲリの存在が観測できなかった時代である。

\begin{figure}[H]
  \centering
  \includegraphics[width=0.8\textwidth]{./section/Kawori/figures/H24Nyugakushiki.jpg}
  \caption{最前列最左端に位置するのが我らのオガワマンであり、その左斜め上に(ポ)が君臨する。マタチチ大先生は最後列右から三番目に存在する。}
\label{fig:H24Nyugakushiki}
\end{figure}

その「カヲリ内部隠遁時代」から3年が経過したタイミングで、入学式を再現しようとする試みが見られた(図\ref{fig:H24NyugakushikiDummy})。
この入学式再現VTRは、後々まで参考にされる非常に重要な実験であり、その実験目的は、一つにはオガワカヲリの存在を明らかにしたい竹田側の意向があったことが知られている。
そのため、まだこの段階でも竹田・又吉はオガワカヲリの存在を知ることはなかったようである。

\begin{figure}[H]
  \centering
  \includegraphics[width=0.8\textwidth]{./section/Kawori/figures/H24NyugakushikiDummy.jpg}
  \caption{最前列最左端に位置するのが我らのオガワマンであるが、しかし入学式の完全再現には至ることはなかった。図\ref{fig:H24Nyugakushiki}と見比べてもらえれば分かる通り、
  オガワマンの立ち位置がタイル一個分ずれてしまっているのである。}
\label{fig:H24NyugakushikiDummy}
\end{figure}

また、入学式再現の後に、入学式爆発実験も行った。
その実験結果を図\ref{fig:H24NyugakushikiDummyAbareru}に示す。
これから分かるように、入学式集合写真を撮影するタイミングで爆発が起きた場合は、人間はこの様に宙を舞うのである。

\begin{figure}[H]
  \centering
  \includegraphics[width=0.8\textwidth]{./section/Kawori/figures/H24NyugakushikiDummyAbareru.jpg}
  \caption{入学式集合写真爆発実験。宙を舞う二人の姿が鮮明に記録されている。}
\label{fig:H24NyugakushikiDummyAbareru}
\end{figure}

\subsection{修士課程入学式}

また、我々が修士課程に入学した段階での写真を図\ref{fig:H28Nyugakushiki}に示す。
竹田・又吉の表情を解析すると、この時点ではオガワカヲリの概念はほとんど固定されていることが分かる。
その存在を強く認識してはいるものの、実際には面会したことがないために、概念としてしか理解できていない苦痛も見て取れる。
この時点ではカヲリという概念は、高度に抽象化された概念であり、その実体をイメージすることは非常に困難であった。
この苦痛が解消され、カヲリという概念が実在する象徴としての存在に昇華するまでにこの時点からさらに二年を要したのである。\par

\begin{figure}[H]
  \centering
  \includegraphics[width=0.8\textwidth]{./section/Kawori/figures/H28Nyugakushiki.jpg}
  \caption{2016年度理学研究科入学式集合写真。}
\label{fig:H28Nyugakushiki}
\end{figure}

\subsection{卒業式において}
以上までで、二種類の入学式について議論した。
それらの比較から、オガワカヲリの概念形成前後での竹田・又吉両名の表情の違いが容易に見て取れる。
また、この時点ではオガワカヲリは、聖なる母として粒子物理研究室のメンバーであれば誰でもその名を知っている存在であった。
これらから分かるようにオガワカヲリは非常に抽象化された、一種の宗教としての性質を持っていたことも心に留めておきたい。
何かに困ったとき、助けてくれるのはカヲリという概念を具象化したワードであり、「かをり、かをり」と続けざまに唱えることにより、
その場が凌げるという効能も持っていた。
\par
そのカヲリであるが、ついに修士課程卒業式のタイミングで、神戸に降臨するとの情報をキャッチすることができた。
今までは概念上の存在であり、オコワを京都から神戸へ運搬していた人物としてしか具体的な情報を持っていなかった我々であるが、ついにその実在化に向け運命の歯車が回り始めたのである。
\par
図\ref{fig:OgawaManKawori}を見ていただきたい。
非常に微笑ましく、我々が入学以来六年間追いかけ続けたカヲリという概念がついに、オガワマンの横で具象化した瞬間である。
写真の女性がカヲリ本人であり、オガワマンをオコワマンとして仕立てていた女性であるのだ。

\begin{figure}[H]
  \centering
  \includegraphics[width=0.8\textwidth]{./section/Kawori/figures/OgawaManKawori.jpg}
  \caption{小川圭将と小川かをりとの夢の共演である。}
\label{fig:OgawaManKawori}
\end{figure}

\par
さらに竹田・又吉両名に加え、粒子物理研究室随一のチャラ男若宮光太郎との、2ショットにも笑顔で答えてくれたのである。
図\ref{fig:KaworiMatayo},\ref{fig:KaworiPo},\ref{fig:KaworiPika}に示す。

\begin{figure}[H]
  \centering
  \includegraphics[width=0.8\textwidth]{./section/Kawori/figures/KaworiMatayo.jpg}
  \caption{又吉清掃員とかをり氏との記念写真。
  何を隠そう、このプールは竹田氏思い出のプールなのである。
  たまに泳ぎに来ていたのである。
  又吉清掃員もまた、高校時代には水泳部という主張をしており、本プールは非常に想い出深い場所であり、
  その場所において伝説の写真が撮影されたわけである。}
\label{fig:KaworiMatayo}
\end{figure}

\begin{figure}[H]
  \centering
  \includegraphics[width=0.5\textwidth]{./section/Kawori/figures/KaworiPo.jpg}
  \caption{かをり概念拡散部隊隊長である竹田氏と、かをり氏との直接対決を記録した記念すべき一枚。
  修士課程卒業には竹田氏の親族は出席していないため、かをり氏の粋な図らいにより竹田母の代役を努めて頂いている。
竹田隊長は日光が眩しいのか、照れくさいのか、それとも今までイジり倒してきた相手をいざ目の前にして半笑いなのか分からないが、その評定はどことなく微笑んでいる。
それに対してかをりは、口角をぐっと上げ、余裕の笑みを浮かべている。
しかし、後ろに回した手は、いついかなるときでも竹田氏を倒すだけの用意ができているということであろうか?\\
どちらかと言えば、この写真の最重要ポイントはポートライナーの駅を出て徒歩10歩の場所で、人通りがものすごくあるドリル目線の環境で撮影されたということであろう。
端から見れば親子写真を撮っている様に感じられるのであろうが、その実全く関係のない二人を、そのかをりの息子が写真を撮っているのである。
もし仮に知り合いに「あの方はお母さん?」と聞かれた場合、返答の選択肢としては説明放棄しか存在しない。
  }
\label{fig:KaworiPo}
\end{figure}

\begin{figure}[H]
  \centering
  \includegraphics[width=0.5\textwidth]{./section/Kawori/figures/KaworiPika.jpg}
  \caption{チャラ男代表のミツ太郎と、かをり氏の直接対決。図\ref{fig:KaworiPo}との違いは、そのチャラさであろう。
  明らかに写真のテーマが真面目さか、チャラさかに二分されるであろう。
  ミツ太郎氏はピースサインを決め、それに対してかをり氏は少し照れているのである。
  }
\label{fig:KaworiPika}
\end{figure}
