\section{まとめ}
Some implications and consequences of the expansion of the universe are examined. In Chapter 1 it is shown that this expansion creates grave difficulties for the Hoyle-Narlikar theory of gravitation. Chapter 2 deals with perturbations of an expanding homogeneous and isotropic universe. The conclusion is reached that galaxies cannot be formed as a result of the growth of perturbations that were initially small. The propagation and absorption of gravitational radiation is also investigated in this approximation. In Chapter 3 gravitational radiation in an expanding universe is examined by a method of asymptotic expansions. The 'peeling off' behaviour and the asymptotic group are derived. Chapter 4 deals with the occurrence of singularities in cosmological models. It is shown that a singularity is inevitable provided that certain very general conditions are satisfied.\\
 A very large scintillating fiber (SciFi) tracking detector for the K2K long baseline neutrino oscillation experiment has been in operation since March, 1999. Track finding efficiency is 98±2 \% for long muon tracks (those which intersect more than 5 fiber planes), and 85±6 \% for short tracks, which were estimated using cosmic-ray muons and a Monte Carlo simulation. The position resolution per layer is about 0.8 mm. The SciFi detector has demonstrated its capability for reconstruction of νμ interactions. The pulse heights for cosmic- ray muons have been stable within 10 \% after one year of operation. In addi- tion, the SciFi detector offers the possibility of performing electron and proton identification.\\
 Christoph is a postdoc at the University of Heidelberg’s physics institute in Germany. He has been working in ATLAS since 2007. His research is currently focussed on the identification of very energetic heavy particles, e.g. top quarks. When he is not looking for physics beyond the Standard model, he likes reading, listening to music, photography, cooking, traveling with his wife, playing board games with friends and sports.

